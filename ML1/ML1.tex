\documentclass{article}
\usepackage{ctex}
\usepackage{amsmath}
\usepackage{graphicx}
\title{机器学习导论作业1}
\begin{document}
\maketitle
\section*{1}\noindent
\subsection*{(1)}\noindent
\begin{equation}\noindent
F_X(x)=\left\{
\begin{aligned}
0 & , & x\le 0 \\
\frac{1}{2}x & , & 0<x\le 1\\
\frac{1}{2}  & , & 1<x\le 2\\
\frac{1}{6}(x+1) &,&2<x\le 5\\
1 &,&x>5
\end{aligned}
\right.
\end{equation}\noindent
\subsection*{(2)}\noindent
$F_Y(y)=P(\frac{1}{X^2}\le y)=P(X>\frac{1}{\sqrt{y}})+P(X<-\frac{1}{\sqrt{y}})(y>0)
\\=F_X(-\frac{1}{\sqrt{y}})+1-F_X(\frac{1}{\sqrt{y}})
=1-F_X(\frac{1}{\sqrt{y}})$\\
\begin{equation}\noindent
F_Y(y)=\left\{
\begin{aligned}
1-\frac{1}{6}(\frac{1}{\sqrt{y}}+1)&,&\frac{1}{25}\le y<\frac{1}{4}\\
\frac{1}{2}&,&\frac{1}{4}\le y<1\\
1-\frac{1}{2\sqrt{y}}&,&y\ge 1\\
0&,& else\\
\end{aligned}
\right.
\end{equation}
$f_Y(y)=F_Y(y)'=f_X(\frac{1}{\sqrt{y}})*(\frac{1}{2})y^{-\frac{3}{2}}$
\begin{equation}\noindent
f_Y(y)=\left\{
\begin{aligned}
\frac{1}{4}y^{-\frac{3}{2}}&,&y>1\\
\frac{1}{12}y^{-\frac{3}{2}}&,&\frac{1}{25}\le y\le\frac{1}{4}\\
0&&else\\
\end{aligned}
\right.
\end{equation}
\subsection*{(3)}\noindent
$E(Z)=\int_{z=0}^{\infty}Pr\{Z\ge z\}dz=\int_{z=0}^{\infty}(1-F_Z(z))dz$\\
$E(Z)=\int_{z=0}^{\infty}zf(z)dz=\int_{z=0}^{\infty}zdF_Z(z)\\
=zF_Z(z)|_{z=\infty}-zF_Z(z)|_{z=0}-\int_{z=0}^{\infty}F_Z(z)dz\\
=z|_{z=_\infty}-\int_{z=0}^{\infty}F_Z(z)dz=\int_{z=0}^{\infty}dz-\int_{z=0}^{\infty}F_Z(z)dz
\\=\int_{z=0}^{\infty}(1-F_Z(z))dz$\\
得证\\
验证:
$E(X)=\int_{0}^{\infty}(1-F_X(x))dx=\int_{0}^{1}(1-\frac{1}{2}x)dx+\int_{1}^{2}\frac{1}{2}dx+\int_{2}^{5}\frac{1}{6}(5-x)dx
\\=\frac{3}{4}+\frac{1}{2}+\frac{5}{2}-\frac{7}{4}=2$\\
$E(X)=\int_{0}^{\infty}xf(x)dx=\int_{0}^{1}\frac{1}{2}xdx+\int_{2}^{5}\frac{1}{6}xdx=\frac{1}{4}+\frac{7}{4}=2$\\
\\
$E(Y)=\int_{0}^{\infty}(1-F_Y(y))dy=\int_{\frac{1}{25}}^{\frac{1}{4}}\frac{1}{6}(\frac{1}{\sqrt{y}}+1)dy+\int_{\frac{1}{4}}^{1}\frac{1}{2}dy+\int_{1}^{\infty}\frac{1}{2}y^{-\frac{1}{2}}dy+\frac{1}{25}
\\=\frac{27}{200}+\frac{3}{8}+\int_{1}^{\infty}\frac{1}{2}y^{-\frac{1}{2}}dy+\frac{1}{25}=\frac{11}{20}+\int_{1}^{\infty}\frac{1}{2}y^{-\frac{1}{2}}dy
\\=y^{\frac{1}{2}}|_{y=\infty}-\frac{9}{20}$\\
$E(Y)=\int_{0}^{\infty}yf_Y(y)dy=\int_{1}^{\infty}\frac{1}{4}y^{-\frac{1}{2}}dy+\int_{\frac{1}{25}}^{\frac{1}{4}}\frac{1}{12}y^{-\frac{1}{2}}dy
\\=\frac{1}{2}y^{\frac{1}{2}}|_{y=\infty}-\frac{9}{20}$\\
E(Y)不存在,无法验证

\section*{2}\noindent
\subsection*{(1)}\noindent
A=rain today,B=rain tomorrow\\
$P(B|A)=\frac{P(AB)}{P(A)}=\frac{0.25}{0.3}=\frac{5}{6}$
\subsection*{(2)}\noindent
$P(G|\neg H)=\frac{P(G\neg H)}{P(\neg H)}=\frac{P(\neg H|G)P(G)}{1-P(H)}=\frac{(1-P(H|G))P(G)}{1-P(H)}\\=\frac{P(G)-P(HG)}{1-P(H)}$
\subsection*{(3)}\noindent
令A代表第一个球是白色,B代表第一个球是黑色,C代表第二个球是白色
$P(C)=P(C|A)P(A)+P(C|B)P(B)=\frac{w}{w+b}*\frac{w+d-1}{w+b+d-1}+\frac{b}{w+b}*\frac{w}{w+b+d-1}=\frac{w}{w+b}$\\
得证
\section*{3}\noindent
\subsection*{(1)}\noindent
由于$x_\perp$是在y上的投影,可以令$x_\perp=(a,\sqrt{3}a)^T$\\
由$xy=x_\perp y$\\
$2\sqrt{3}=a+3a\rightarrow a=\frac{\sqrt{3}}{2}$\\
$x_\perp=(\frac{\sqrt{3}}{2},\frac{3}{2})^T$
\subsection*{(2)}\noindent
$x-x_\perp=(\frac{\sqrt{3}}{2},-\frac{1}{2})^T$\\
$y(x-x_\perp)=0$\\
因此$y\perp(x-x_\perp)$
\subsection*{(3)}\noindent
$||x-x_\perp||=1$\\
$||x-\lambda y||=|(\sqrt{3}-\lambda,1-\sqrt{3}\lambda)^T|=\sqrt{3-2\sqrt{3}\lambda+\lambda^2+1-2\sqrt{3}\lambda+3\lambda^2}
\\=2\sqrt{\lambda^2-\sqrt{3}\lambda+1}=2\sqrt{(\lambda-\frac{\sqrt{3}}{2})^2+\frac{1}{4}}\ge 1$\\
得证
\section*{4}\noindent
原假设$H_0:\mu_0\le0.5$\\
令$T=\frac{\overline{X}-\mu_0}{S/\sqrt{n}}~t(n-1)$\\
拒绝域为$T=\frac{\overline{X}-\mu_0}{S/\sqrt{n}}>t_\alpha(n-1)$\\
$S=\sqrt{\frac{1}{49}(0.7^2*15+0.3^2*35)}=0.463$\\
而T=$\frac{0.7-0.5}{0.463/7}=3.024,t_0.05(49)=1.676$\\
T落在拒绝域内,原假设被拒绝\\
因此可以认为$\mu>0.5$,也即硬币不均匀偏向人头的一边
\section*{5}
\subsection*{(1)}
\includegraphics[width=8cm,height=6cm]{roc.jpg}\\
$AUC_{C_1}=0.75,AUC_{C_2}=0.4375$
\subsection*{(2)}
$C_1:$
\begin{tabular}{|c|c|c|}\hline
真实类别/预测类别&1&0\\\hline
1&3&1\\\hline
0&1&3\\\hline
\end{tabular}\\
$P=\frac{TP}{TP+FP}=\frac{3}{4},R=\frac{TP}{TP+FN}=\frac{3}{4}\\
F_1=\frac{2PR}{P+R}=\frac{3}{4}$\\
$C_2:$
\begin{tabular}{|c|c|c|}\hline
	真实类别/预测类别&1&0\\\hline
	1&3&1\\\hline
	0&3&1\\\hline
\end{tabular}\\
$P=\frac{TP}{TP+FP}=\frac{1}{2},R=\frac{TP}{TP+FN}=\frac{3}{4}\\
F_1=\frac{2PR}{P+R}=\frac{3}{5}$
\end{document}